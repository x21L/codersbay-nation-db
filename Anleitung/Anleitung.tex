\documentclass[12pt, a4paper]{article}
\usepackage[default,scale=0.95]{opensans}
\usepackage[ngerman]{babel}
% \usepackage[T1]{fontenc}
\usepackage[utf8]{inputenc}
\usepackage{hyperref}
\usepackage{minted}
\usepackage{fullpage}

\title{Datenbanksetup Anleitung}
\author{Lukas Wais}
\date{\today}
\begin{document}
\maketitle

Hier wird beschrieben wie die Datenbank aufgesetzten und betrieben wird. \\
Außerdem, werden häufig gestellte Fragen beantwortet.

\section*{Voraussetzungen}
\begin{enumerate}
    \item Installation von podman
          \begin{itemize}
              \item Die Kommandozeilenversion ist ausreichend. Podman Desktop wird nicht benötigt.
              \item \href{https://github.com/containers/podman/blob/main/docs/tutorials/podman-for-windows.md}{Windows} Anleitung
              \item \href{https://podman.io/getting-started/installation}{MacOS + Linux} Anleitung
          \end{itemize}
    \item \href{https://www.python.org/downloads/}{Python3}, wird vor allem für \mintinline{shell}{podman-compose} benötigt.
          \begin{itemize}
              \item Es wird nicht empfohlen die Version aus dem Windows Store zu installieren.
              \item Für Windows nicht vergessen \textbf{Python zum Pfad hinzuzufügen!} Einfach die Checkbox bei der Installation wählen.
          \end{itemize}
    \item \href{https://github.com/containers/podman-compose}{podman-compose}
          \begin{itemize}
              \item Die Installation erfolgt mit pip. Darum ist im Terminal folgender Befehl auszuführen: \mintinline{shell}{pip3 install podman-compose}
          \end{itemize}
    \item \href{https://git-scm.com/}{Git} zum klonen des Repositories
    \item  Keine Voraussetzung, aber empfohlen wird auch der \href{https://github.com/microsoft/terminal}{Windows Terminal}
\end{enumerate}

\newpage

\section*{Installation}
\begin{enumerate}
    \item Repository klonen und im Terminal öffnen. Die Ausgabe sollte wie folgt aussehen:
\begin{minted}[frame=single, fontsize=\small]{shell}
    Mode                 LastWriteTime         Length Name
----                 -------------         ------ ----
d----          21/10/2022    11:10                Anleitung
-----          21/10/2022    09:48             79 Containerfile
-----          21/10/2022    09:48            290 docker-compose.yml
-----          21/10/2022    09:48         326276 init.sql
-----          21/10/2022    09:48           1083 LICENSE
-----          21/10/2022    10:54            221 README.md
\end{minted}

    \item \mintinline{shell}{podman-compose up -d} auszuführen
    \item Warten bis das Setup abgeschlossen ist
\end{enumerate}

\section*{Betrieb}
\begin{itemize}
    \item podman starten (notwending nach Computer Neustart)
    \item[] \mintinline{shell}{podman machine start}
    \item Zum starten der DB benötigt man den Containernamen. Den findet man mit \mintinline{shell}{podman ps -a} raus.
\begin{minted}[frame=single, fontsize=\footnotesize]{shell}
PS C:\Users\lukas> podman ps -a
CONTAINER ID  ... NAMES
e1fa7e625099  ... sleepy_meninsky
4901ccb6c874  ... codersbay-nation-db
\end{minted}

\item \textbf{Datenbank starten} \mintinline{shell}{podman start codersbay-nation-db} \\
Der Name kann/wird unterschiedlich sein!

\item Datenbank stoppen \mintinline{shell}{podman start codersbay-nation-db}
\item podman stoppen \mintinline{shell}{podman machine stop}
\end{itemize}

\newpage

\section*{FAQ}
\begin{itemize}
    \item[$\dagger$] Es kann keine Verbindung zur VM hergestellt werden gibt \mintinline{shell}{podman machine init}  aus, dass es bereits eine Maschine gibt hilft es die VM neu zu installieren.
\end{itemize}
\begin{minted}{shell}
    podman machine rm
\end{minted}

\begin{itemize}
    \item[] Anschließend muss sie neu installiert werden, das geschieht mit
\end{itemize}
\begin{minted}{shell}
    podman machine init
\end{minted}
\end{document}